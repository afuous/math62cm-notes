\documentclass[10pt]{article}
\usepackage{amsmath}
\usepackage{amssymb}
\usepackage{amsthm}
\usepackage{mathtools}
\usepackage{enumerate}
\usepackage[margin=1in]{geometry}
\setlength{\parindent}{0pt}

\newcommand{\gap}{\vspace{3mm}}
\newcommand{\bb}[1]{\mathbb{#1}}
\newcommand{\abs}[1]{\left\lvert#1\right\rvert}
\newcommand{\inner}[2]{\langle #1, #2 \rangle}
\newcommand{\de}{\mathrm{d}}
\newcommand{\frakx}{\mathfrak{X}}
\newcommand{\partials}[2]{\frac{\partial #1}{\partial #2}}

\newcommand{\p}[1]{\left( #1 \right)}
\renewcommand{\c}[1]{\left\{ #1 \right\}}
\newcommand{\f}[2]{\frac{#1}{#2}}

\newcommand{\R}{\mathbb{R}}
\newcommand{\Z}{\mathbb{Z}}
\newcommand{\N}{\mathbb{N}}
\newcommand{\Q}{\mathbb{Q}}
\newcommand{\C}{\mathbb{C}}

\newcommand\m[1]{\begin{pmatrix}#1\end{pmatrix}}
\newcommand\mb[1]{\begin{bmatrix}#1\end{bmatrix}}

\begin{document}

	\section*{Definitions}

	Let $V$ be a vector space over $\bb{R}$. The dual space of $V$ is $V^*$, the set of linear functions $V \to \bb{R}$. Let $0 \leq k < \dim(V)$. The tensor product $(V^*)^{\otimes k}$ is the set of k-linear functions $V^k \to \bb{R}$. The exterior k-forms $\Lambda^k(V^*)$ are a subset of $(V^*)^{\otimes k}$, specifically the asymmetric functions.

	\gap
	Let $M$ be a smooth manifold. Then $T_p M$ is the tangent plane of $M$ at the point $p \in M$. Also, $TM$ is informally used as a function $p \mapsto T_p M$. The dual space $(T_p M)^*$ is written $T_p^* M$. Similarly, $T^* M$ refers to $p \mapsto T_p^* M$. (Apparently $TM$ is called the tangent bundle and $T^* M$ is called the cotangent bundle.) Note that $T_p \bb{R}^n$ is $\bb{R}^n$ itself for any $p \in \bb{R}^n$.

	\gap
	Many of these linear algebraic objects can be parameterized by a point on a manifold. For example, vector spaces are generalized by smooth vector fields. For a smooth manifold $M$, define $\frakx(M)$ as the set of smooth functions that take each point $p \in M$ to a vector in $T_p M$. Restricting to a single point $p$, we get a vector space ($T_p M$).

	Differential forms generalize exterior forms in the same way. For $0 \leq k < \dim(V)$, $\Omega^k(M)$ is the set of "smooth" functions that take each point $p \in M$ to a $k$-form in $\Lambda^k(T_p^* M)$. Restricting to a single point $p$, all we get a space of $k$-forms ($\Lambda^k(T_p^* M)$). The differential forms $\Omega^k(M)$ were defined in class as $\{\sum_I a_I \de x_I \mid a_I \text{ smooth}\}$. This is essentially equivalent to the definition here, except "smooth" is in quotes since we never defined what it means for a function like this to be smooth, but it is the same idea. Note that $\Omega^0(M)$ is just $C^\infty(M, \bb{R})$, the set of all smooth functions $M \to \bb{R}$.

	\section*{$\de$}
	The symbol $\de$ has many meanings. Suppose $f : M \to N$ is a smooth function between smooth manifolds $M$ and $N$. Then for $p \in M$, $\de f|_p : T_p M \to T_{f(p)} N$ is linear. Explicitly, with $v \in T_p M$, $\de f|_p(v) = \sum \partials{f}{x_i} v_i$. % TODO: this last part sucks

	\subsection*{$\de : \Omega^k \to \Omega^{k+1}$}

	\section*{Pullbacks}
	Let $f : M \to N$ be smooth and $\omega \in \Omega^k(N)$, so for $q \in N$, $\omega|_q \in \Lambda^k(T_q^* N)$. We want the pullback to be $f^* \omega \in \Omega^k(M)$, so for $p \in M$, $(f^* \omega)|_p \in \Lambda^k(T_p^* M)$. Using the fact that exterior $k$-forms are $k$-linear functions, and for $p \in M$ and $v_1, \ldots, v_k \in T_p M$, the pullback is defined as
	\[ ((f^* \omega)|_p)(v_1, \ldots, v_k) = (\omega|_{f(p)})(\de f|_p(v_1), \ldots, \de f|_p(v_k)). \]
	The reason $\de f|_p$ is involved is because it maps from $T_p M$ to $T_{f(p)} N$. In fact, the pullback $(f^* \omega)|_p$ is really the dual of the linear map $\de f|_p$, namely $(\de f|_p)^* (\omega|_{f(p)})$. Less precisely, pullbacks can be written by excluding all mentions of $p$ and making the pointwise definition implicit:
	\[ f^* \omega(v_1, \ldots, v_k) = \omega(\de f(v_1), \ldots, \de f(v_k)). \]

\end{document}
